
\documentclass{studrep}

\RequirePackage{tabularx}
% \RequirePackage{graphicx}

\graphicspath{{pics/}}

% \usepackage{showframe}

% pygmetize
\RequirePackage{minted}
\RequirePackage{indentfirst}
\usemintedstyle{tango} % bw

% \setmainfont{Times New Roman}

\setminted{autogobble,mathescape,linenos=true,fontsize=\footnotesize}
\begin{document}
\thispagestyle{empty}
\begin{center}
Министерство науки и высшего образования Российской Федерации

Федеральное государственное бюджетное образовательное учреждение\\ высшего образования

«Иркутский государственный университет»\\
(ФГБОУ ВО «ИГУ»)

Институт математики и информационных технологий

Кафедра информационных технологий
\end{center}

\vfill
\begin{center}
  \textbf{\large ОТЧЕТ}

  по научно"=исследовательской работе
  % по производственной практике
  % по преддипломной практике   % и т.п.
  %\textbf{ ВЫПУСКНАЯ КВАЛИФИКАЦИОННАЯ РАБОТА\\
%БАКАЛАВРА}
\vspace{1em}

% по направлению 02.03.03 -- Математическое обеспечение и администрирование информационных систем

% профиль <<общий>>

\vspace{2em}
ТЕМА БОЛЬШИМИ БУКВАМИ

\end{center}
\vfill

\noindent\begin{tabularx}{\textwidth} {
  >{\raggedright\arraybackslash}X
  >{\raggedright}X }
&

Студента \_ курса очного отделения\\
группы 02\_4\_-ДБ\\
Фамилия Имя Отчество\\[2em]

Руководитель:\\
к.~т.~н., доцент\\
\underline{\hspace{3cm}} Иванов Иван Иванович\\[2em]

Защищен с оценкой\\[1em] \underline{\hspace{3cm}}
%Допущена к защите\\
%Зав.каф., к.~т.~н., доцент\\
%\underline{\hspace{3cm}} Черкашин Евгений\\ \hspace{3cm} Александрович\\[2em]

\end{tabularx}
\vfill
\begin{center}
  Иркутск 2023
\end{center}
\clearpage

\tableofcontents

\chapter*{ВВЕДЕНИЕ}
\label{chap:intro}
% TODO: Add contents line

Реализация компиляторов языков программирования -- одно из основных направлений в области системного программирования, включаещего разработку трансляторов (в общем смысле, т.е. и компиляторов и интерпретаторов).  Трансляторы языков проргаммировая относятся к системам порождающего программирования (ПП), т.е. программным системам, задача которых создать исходны код или какой-либо другой объект по некоторому описанию, модели, исходному информационному объекту.  Применение ПП предполагает, что исходный информационый объект меняется достаточно редко, поэтому имеет смысл повысить производительность целевой системы за счет представления предварительного анализа объекта в виде последовательности инструкций целевого вычислителя, реализующих уже результат анализа.  Сама процедура анализа выполняется один раз транслятором.

Разработка трансляторов позволяет решать следующие задачи:
\begin{itemize}
\item Разрабатывать новые системы программирования;
\item Переносить существующий программный кодя языка высокого уровня на новые вычислительные платформы, например, микроконтроллеры;
\item Разрабатывать языки описания предметной области, представляющие объекты предметной области в удобном для пользователя виде, а, затем, преобразовывать описание в какой-либо другой язык для решения задачи;
\item Проводить исследования в области системного программирования и защиты информации;
\item Развивать математические аспекты теории формальных языков.
\end{itemize}

\textbf{Целью} данной курсовой работы является исследование компилятора \verb|LDC| версии 2.0 языка программирования высокого уровня \verb|D|.

В курсовой работе решены следующие \textbf{звдачи}:
\begin{itemize}
\item Изучен язык программирования \verb|D|, параметры компилятора \verb|LDC2| и система сборки пакетов \verb|dub|.
\item Создана программа, микросервис веб, вычисляющий факториал передаваемого значения.
\item Настроена система сборки проекта, реализована сборка.
\item Исследовано использование аргуметов компилятора в процессе сборки проекта.
\item Осуществлен запуск компилятора для исследуемого модуля с целью трансляции исходного кода в промежуточное представление (ПП).
\item Выявлен и проанализирован текст ПП, относящийся к исследуемому методу.
\end{itemize}

\chapter{Язык и среда программирования LDC}



\chapter{Проектирование и реализация микросервиса}



\begin{minted}{text}
  stud@sysrescue:~/projects/webapp$ dub run --compiler=ldc2
    Starting Performing "debug" build using ldc2 for x86_64.
  Up-to-date hunt 1.7.17: target for configuration [library] is up to date.
  Up-to-date hunt-extra 1.2.3: target for configuration [library] is up to date.
  Up-to-date hunt-net 0.7.1: target for configuration [default] is up to date.
  Up-to-date hunt-http 0.8.2: target for configuration [default] is up to date.
  Up-to-date protobuf 0.6.2: target for configuration [protobuf] is up to date.
  Up-to-date grpc 0.5.0-beta.2: target for configuration [library] is up to date.
  Up-to-date hunt-redis 1.4.1: target for configuration [library] is up to date.
  Up-to-date hunt-cache 0.10.1: target for configuration [library] is up to date.
  Up-to-date hunt-console 0.4.0: target for configuration [hunt-console] is up to date.
  Up-to-date hunt-sql 1.6.0: target for configuration [library] is up to date.
  Up-to-date hunt-database 2.3.6: target for configuration [default] is up to date.
  Up-to-date hunt-validation 0.5.0: target for configuration [library] is up to date.
  Up-to-date hunt-entity 2.8.1: target for configuration [library] is up to date.
  Up-to-date hunt-openssl 1.0.5: target for configuration [library] is up to date.
  Up-to-date hunt-jwt 0.2.0-beta.4: target for configuration [library] is up to date.
  Up-to-date hunt-security 0.6.0: target for configuration [library] is up to date.
  Up-to-date hunt-shiro 1.3.1: target for configuration [library] is up to date.
  Up-to-date poodinis 8.1.3: target for configuration [library] is up to date.
  Up-to-date hunt-framework 3.4.6: target for configuration [library] is up to date.
  Up-to-date webapp ~master: target for configuration [application] is up to date.
    Finished To force a rebuild of up-to-date targets, run again with --force
     Running webapp


 ___  ___     ___  ___     ________      _________
|\  \|\  \   |\  \|\  \   |\   ___  \   |\___   ___\     Hunt Framework 3.4.6
\ \  \\\  \  \ \  \\\  \  \ \  \\ \  \  \|___ \  \_|
 \ \   __  \  \ \  \\\  \  \ \  \\ \  \      \ \  \      Listening: 0.0.0.0:8080
  \ \  \ \  \  \ \  \\\  \  \ \  \\ \  \      \ \  \     TLS: Disabled
   \ \__\ \__\  \ \_______\  \ \__\\ \__\      \ \__\
    \|__|\|__|   \|_______|   \|__| \|__|       \|__|    https://www.huntframework.com


Try to browse http://0.0.0.0:8080
\end{minted}


\begin{minted}{text}
stud@sysrescue:~/projects/webapp/source/app/controller$ ps -FC webapp
UID          PID    PPID  C    SZ   RSS PSR STIME TTY          TIME CMD
stud       18551   18544  0 163131 37192  0 17:25 pts/1    00:00:00 /home/stud/projects/webapp/webapp
\end{minted}

\begin{minted}{bash}
stud@sysrescue:~/projects/webapp/source/app/controller$ curl http://localhost:8080/api/fact/9
{"result":"362880"}
stud@sysrescue:~/projects/webapp/source/app/controller$ curl http://localhost:8080/api/test
{"currtime":"2023-11-07T17:25:59.8807696"}
stud@sysrescue:~/projects/webapp/source/app/controller$ curl http://localhost:8080/api/echo/message-to-test
{"echo":"message-to-test"}
\end{minted}

\begin{minted}{html}
  <!--
$ stud@sysrescue:~/projects/webapp/source/app/controller$ curl http://localhost:8080/api/fact/9e
  -->
<!doctype html>
<html lang="en">
    <meta charset="utf-8">
    <title>404 Not Found</title>
    <meta name="viewport" content="width=device-width, initial-scale=1">
    <style>

        * {
            line-height: 3;
            margin: 0;
        }

        html {
            color: #888;
            display: table;
            font-family: sans-serif;
            height: 100%;
            text-align: center;
            width: 100%;
        }

        body {
            display: table-cell;
            vertical-align: middle;
            margin: 2em auto;
        }

        h1 {
            color: #555;
            font-size: 2em;
            font-weight: 400;
        }

        p {
            margin: 0 auto;
            width: 90%;
        }

    </style>
</head>
<body>
    <h1>404 Not Found</h1>
    <p>Sorry!! Unable to complete your request :(</p>

</body>
</html>
\end{minted}

\chapter{Исследование скомпилированного кода}

\begin{minted}{d}
module app.controller.ApiController;

import hunt.framework;
import std.json : JSONValue;
import std.stdio;
import std.conv;

class ApiController : Controller
{
  mixin MakeController;

  // . . . . . . . . . . . . . .

  @Action
  JsonResponse fact(string n) {
    JSONValue js;
    auto res = fact(to!int(n));
    js["result"] = to!string(res);
    auto resp = new JsonResponse(js);
    return resp;
  }

  int fact(int n) {
    if (n==0) return 1;
    if (n==1) return 1;
    return n*fact(n-1);
  }
}
\end{minted}

Метод \verb-int fact(int n)- конвертирован в промежуточное представление, в результате получен следующий текст (Листинг~\ref{lis:fact-llvn}).  Моменты, представляющие интерес, прокомментированны на русском языке.

\begin{listing}[H]
\caption{Промежуточное представление метода fact}
\begin{minted}{llvm}
; [#uses = 1]
; Function Attrs: uwtable
define i32 @_D3app10controller13ApiControllerQp4factMFiZi(%app.controller.ApiController.ApiController* nonnull %.this_arg, i32 %n_arg) #0 {
  ; Длинное название функции обусловлено вхождением исходного метода
  ; в контексты пакета app.controller и модуль ApiController.
  ; В метод в качестве первого аргумента передается указатель
  ; на экземпляр класса ApiController - this.
  %this = alloca %app.controller.ApiController.ApiController*, align 8 ; [#uses = 3, size/byte = 8]
  %n = alloca i32, align 4                        ; [#uses = 5, size/byte = 4]
  store %app.controller.ApiController.ApiController* %.this_arg, %app.controller.ApiController.ApiController** %this, align 8
  store i32 %n_arg, i32* %n, align 4
  %1 = load i32, i32* %n, align 4                 ; [#uses = 1]
  %2 = icmp eq i32 %1, 0                          ; [#uses = 1]
  br i1 %2, label %if, label %endif

if:                                               ; preds = %0
  ret i32 1

dummy.afterreturn:                                ; No predecessors!
  br label %endif

endif:                                            ; preds = %dummy.afterreturn, %0
  %3 = load i32, i32* %n, align 4                 ; [#uses = 1]
  %4 = icmp eq i32 %3, 1                          ; [#uses = 1]
  br i1 %4, label %if1, label %endif2

if1:                                              ; preds = %endif
  ret i32 1

dummy.afterreturn3:                               ; No predecessors!
  br label %endif2

endif2:                                           ; preds = %dummy.afterreturn3, %endif
  %5 = load i32, i32* %n, align 4                 ; [#uses = 1]
  %6 = load %app.controller.ApiController.ApiController*, %app.controller.ApiController.ApiController** %this, align 8 ; [#uses = 1]
  %7 = getelementptr inbounds %app.controller.ApiController.ApiController, %app.controller.ApiController.ApiController* %6, i32 0, i32 0 ; [#uses = 1, type = [38 x i8*]**]
  %8 = load [38 x i8*]*, [38 x i8*]** %7, align 8 ; [#uses = 1]
  %"fact@vtbl" = getelementptr inbounds [38 x i8*], [38 x i8*]* %8, i32 0, i32 36 ; [#uses = 1, type = i8**]
  %9 = load i8*, i8** %"fact@vtbl", align 8       ; [#uses = 1]
  %fact = bitcast i8* %9 to i32 (%app.controller.ApiController.ApiController*, i32)* ; [#uses = 1]
  %10 = load %app.controller.ApiController.ApiController*, %app.controller.ApiController.ApiController** %this, align 8 ; [#uses = 1]
  %11 = load i32, i32* %n, align 4                ; [#uses = 1]
  %12 = sub i32 %11, 1                            ; [#uses = 1]
  %13 = call i32 %fact(%app.controller.ApiController.ApiController* nonnull %10, i32 %12) ; [#uses = 1]
  %14 = mul i32 %5, %13                           ; [#uses = 1]
  ret i32 %14
}
\end{minted}
\end{listing}


\chapter*{ЗАКЛЮЧЕНИЕ}

В таблице~\ref{tbl:ex} представлены результаты сравнения производительности\ldots

\begin{table}[hb]
  \caption{Пример таблицы}\label{tbl:ex}
  \centering
\begin{tabularx}{1\textwidth} {
  | >{\raggedright\arraybackslash}X
  | >{\centering\arraybackslash}X
  | >{\raggedleft\arraybackslash}X | }
 \hline
 item 11 & item 12 & item 13 \\
 \hline
 item 21  & item 22  & item 23  \\
\hline
\end{tabularx}
\end{table}

\begin{thebibliography}{99}
\bibitem{bratko90} И.~Братко. Язык программирования Пролог для искусственного интеллекта. М\;:~Наука. 1990. 310~c.
\bibitem{b2} DeRidder J.L. The immediate prospects for the application of ontologies in digital libraries // Knowledge Organization -- 2007. -- Vol. 34, No. 4. P.~227-246.
\bibitem{b2}  U.S. National Library of Medicine. Fact sheet: UMLS Metathesaurus\;:\;[текст]~/ National Institutes of Health, 2006-2013. -- URL:
\url{http://www.nlm.nih.gov/pubs/factsheets/umlsmeta.html} (дата обращения: 2014-12-09).
\bibitem{b2}  U.S. National Library of Medicine. Fact sheet: Unfied Medical Language System\;:\;[текст]~/ National Institutes of Health, 2006--2013. -- URL:\url{http://www.nlm.nih.gov/pubs/factsheets/umls.html} (дата обращения: 2009-12-09).
\bibitem{b3}  Антопольский А.Б., Белоозеров В.Н. Процедура формирования макротезауруса политематических информационных систем // Классификация и кодирование. -- 1976. -- N 1 (57). -- С.~25--29.
\bibitem{b4}  Белоозеров В.Н., Федосимов В.И. Место макротезауруса в лингвистическом обеспечении сети органов научнотехнической информации // Проблемы информационных систем. -- 1986. -- N 1. -- С.~6--10.
\bibitem{b5}  Использование и ведение макротезауруса ГАСНТИ:Методические рекомендации / ГКНТСССР -- М., 1983. -- 12~с.
\bibitem{b6}  Nuovo soggettario: guidaalsistemaitaliano di indicizzazione per soggetto, prototipo del thesaurus\;:\;[рецензия]~// Knowledge Organization. -- 2007. -- Vol. 34, N 1. -- P.~58--60.
\bibitem{b7}  ГОСТ 7.25-2001 СИБИД. Тезаурус информационно-поисковый одноязычный. Правила разработки, структура, состав и форма представления. -- М., 2002. -- 16~с.
\bibitem{b8}  Nanoscale Science and Technology Supplement: Collection ofapplicable terms fromPACS 2008\;:\;[текст]~// PACS 2010 Regular Eddition /~AIP Publishing. -- URL: \url{http://www.aip.org/publishing/pacs/nano-supplement} (дата обращения: 2014-12-09).
\bibitem{b9}  Смирнова О.В. Методикасоставления индексов УДК // Научно-техническая информация. Сер.~1. -- 2008. -- N 8. -- С.~7--8.
\bibitem{b10}  Индексирование фундаментальных научных направлений кодами информационных классификаций УДК /~О.А.~Антошкова, Т.С.~Астахова, В.Н.~Белоозеров и др.; под ред.акад. Ю.М.~Арского. -- М., 2010. -- 322~с.
\bibitem{b11}  Рубрикатор как инструмент информационной навигации /~Р.С.~Гиляревский, А.В.~Шапкин, В.Н.~Белоозеров. - СПб.\;: Профессия, 2008. -- 352~с.
\bibitem{b12}  Рубрикатор научно-технической информации по нанотехнологиям и наноматериалам / РНЦ "Курчатовский институт",
ФГУ ГНИИ ИТТ "Информика", Национальный электронно-информационный консорциум (НЭИКОН), Всероссийский институт научной и технической информации (ВИНИТИРАН). -- М., 2009. -- 75~с.
\bibitem{b13}  Рубрикатор по нанонауке и нанотехнологиям\;:\;[сайт]~-- URL:\url{http/www.rubric.neicon.ru} (дата обращения: 12-02-2022)
\end{thebibliography}

\appendices

\chapter{Исходный код программ}
\chapter{Документация разработчика}

\end{document}




%%% Local Variables:
%%% mode: latex
%%% TeX-master: t
%%% End:
